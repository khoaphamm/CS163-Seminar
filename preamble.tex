\usepackage[utf8]{vietnam}
\usepackage[english]{babel}
\usepackage[usenames,svgnames,dvipsnames]{xcolor}
\usepackage{tocloft}
\usepackage{booktabs}
\usepackage{siunitx}



\usepackage{tocloft}

% Change the title of the table of contents
\renewcommand{\contentsname}{Table of Contents}

% Change the font of the section titles
\renewcommand{\cftsecfont}{\bfseries}

% Change the font of the page numbers
\renewcommand{\cftsecpagefont}{\bfseries}

% Add a dot after the section numbers
\renewcommand{\cftsecaftersnum}{.}

% Change the indentation of the subsections
\setlength{\cftsubsecindent}{1cm}

%Fill with dots
\renewcommand{\cftsecleader}{\cftdotfill{\cftdotsep}}





\usepackage{amsmath}
\usepackage{listings}
\usepackage{xcoffins,calc}
\usepackage{amsthm}
\usepackage{amsfonts}
\usepackage{amssymb}
\usepackage{graphicx}
\usepackage{yhmath}
\usepackage{hyperref}
\usepackage[left=3.00cm, right=2.00cm, top=2.00cm, bottom=2.00cm]{geometry}
%\usepackage[amsthm]{ntheorem}
%\renewcommand\thechapter{\Roman{chapter}}
\renewcommand\thesection{\arabic{section}}
\theoremstyle{definition}
\newtheorem{bt}{Bài}

%\theoremstyle{nonumberplain}
%\theoremstyle{break}
%\numberwithin{equation}{chapter}
\usepackage{ esint }
\usepackage{array}
\usepackage{enumerate} 
%\usepackage{enumitem}
\usepackage{array}
\usepackage{longtable}
\usepackage{tikz}
\usetikzlibrary{graphs,graphs.standard}
\usepackage{tkz-euclide,amsmath}
\usepackage{ifthen}
\usepackage{mathrsfs}
\usetikzlibrary{arrows}
\pagestyle{empty}
\pagestyle{plain}
\usepackage{ mathrsfs } %chữ in hoa
%  \usepackage{pageborder} %khung bìa
\usepackage[utf8]{inputenc} 


\usetikzlibrary{arrows,%
                shapes,positioning}
                

\usetikzlibrary{calc}
\usepackage{longfbox}
\usepackage{indentfirst}
\usepackage{pgfplots}
\pgfplotsset{compat=1.15}
\usepackage{mathrsfs}
\usepackage{tikz}
\usepackage[most]{tcolorbox}
\usepackage{sectsty}


\usepackage[Glenn]{fncychap}
\let\svlim\lim\def\lim{\svlim\limits}
\newcommand\N{\ensuremath{\mathbb{N}}}
\newcommand\R{\ensuremath{\mathbb{R}}}
\newcommand\Z{\ensuremath{\mathbb{Z}}}
\renewcommand\O{\ensuremath{\emptyset}}
\newcommand\Q{\ensuremath{\mathbb{Q}}}
\newcommand\C{\ensuremath{\mathbb{C}}}
\renewcommand\P{\ensuremath{\mathcal{P}}}
\newcommand\V{\ensuremath{\nu}}

\let\implies\Rightarrow
\let\impliedby\Leftarrow
\let\iff\Leftrightarrow
\let\epsilon\varepsilon


% Setup for some boxes used in courses

\definecolor{gree}{RGB}{7, 135, 44}
%%%%%%%%%%%%%%%%%%%%%%%%%%%%%%%%%%%%%%%%%%%%%%%%%%%%

\newtcolorbox{claim}{
    enhanced,
    boxrule=0pt,frame hidden,
    borderline west={2pt}{0pt}{green!50!black},
    colback=green!10!white,
    sharp corners
}

\newtcolorbox{remark}{
    enhanced,
    boxrule=0pt,frame hidden,
    borderline west={4pt}{0pt}{black},
    colback=black!10!white,
    sharp corners
}

\newcommand{\Claim}{\textbf{\textcolor{green!50!black}{Claim -- }}}

\newcommand{\Remark}{\textbf{Why? -- }}

\usepackage{thmtools}
\usepackage[framemethod=TikZ]{mdframed}

\theoremstyle{definition}
\mdfdefinestyle{mdbluebox}{%
    roundcorner = 10pt,
    linewidth=1pt,
    skipabove=12pt,
    innerbottommargin=9pt,
    skipbelow=2pt,
    nobreak=true,
    linecolor=blue,
    backgroundcolor=TealBlue!5,
}
\declaretheoremstyle[
headfont=\sffamily\bfseries\color{MidnightBlue},
mdframed={style=mdbluebox},
headpunct={\\[3pt]},
postheadspace={0pt}
]{thmbluebox}

\declaretheorem[%
style=thmbluebox,name=Theorem,numberwithin=section]{theorem}
\declaretheorem[style=thmbluebox,name=Bổ đề,sibling=theorem]{lemma}
\declaretheorem[style=thmbluebox,name=Proposition,sibling=theorem]{proposition}
\declaretheorem[style=thmbluebox,name=Corollary,sibling=theorem]{corollary}

%%%%%%%%%%%%%%%%%%%%%%%%%%%%%%%%%%%%%%%%%%%%%%%%%%%%%

\usepackage{import}
\usepackage{xifthen}
\usepackage{pdfpages}
\usepackage{transparent}

\newcommand{\incfig}[1]{%
    \def\svgwidth{\columnwidth}
    \import{./figures/}{#1.pdf}
}
\newcommand{\kt}{\hfill{$\square$}}

\definecolor{atomkeywords}{rgb}{0.55, 0.47, 0.68}
\definecolor{atomstrings}{rgb}{0.78, 0.63, 0.45}
\definecolor{atomnumbers}{rgb}{0.5,0.5,0.5}
\definecolor{atomcomments}{rgb}{0.5, 0.5, 0.5}
\definecolor{atomoperators}{rgb}{0.4, 0.4, 0.4}
\definecolor{atombackground}{rgb}{0.95, 0.95, 0.95}
\definecolor{atomfunctions}{rgb}{0.4, 0.6, 0.8}

\lstdefinestyle{atomOneLight}{
    language=python,
    backgroundcolor=\color{atombackground},   
    commentstyle=\color{atomcomments},
    keywordstyle=\color{atomkeywords},
    numberstyle=\tiny\color{atomnumbers},
    stringstyle=\color{atomstrings},
    basicstyle=\ttfamily\footnotesize,
    breakatwhitespace=false,         
    breaklines=true,                 
    captionpos=b,                    
    keepspaces=true,                 
    numbers=left,                    
    numbersep=5pt,                  
    showspaces=false,                
    showstringspaces=false,
    showtabs=false,                  
    tabsize=2,
    classoffset=1,
    %%The below keywords will be highlighted in the code snippet.
    morekeywords={factorial, is_even, is_odd, tail_factorial, knapSack, fibonacci, tree, max, visit},
    keywordstyle=\color{atomfunctions},
    classoffset=0
}

\lstset{style=atomOneLight}
